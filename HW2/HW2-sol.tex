\documentclass{article}[12pt]
\usepackage{enumerate}
\usepackage{qtree}
\usepackage[localise]{xepersian}

\setdigitfont[Scale=1]{PGaramond}
\setlatintextfont[Scale=1]{XB Niloofar}
\settextfont{XB Tabriz}

\شروع{نوشتار}
\شروع{وسط‌چین}
به نام خدا \\

تمرین سری دوم
\پایان{وسط‌چین}
\شروع{شمارش}
\فقره	
مشکل تعریف گرامر در زبان 
\lr{C}
ناتوانی تعریف کامنت در کامنت می‌باشد چه به صورت چند خطی و چه به صورت تک خطی. راه حل آن نیز استفاده از 
\lr{escape character}
می‌باشد. به این صورت که در صورت خواستن کامنت در کامنت کردن می‌باشد برای تعریف کاراکتر \ در کامنت باید به صورت /\ از آن‌ها استفاده کرد.
\فقره سوال ۲
\فقره
\شروع{شمارش}
\فقره 
در هنگام تعریف متغیر یک ردیف جدید در جدول نمادها ایجاد می‌شود و شناسه آن در ردیف جدول نمادها برگردانه می‌شود. این ردیف شامل 
\lr{symbol name, type, attribute}
است.
\فقره
بررسی وجود یا عدم وجود متغیر مربوطه در جدول نمادها سپس برگرداندن شناسه‌ی مربوط به ردیف آن درصورت وجود.
\فقره
اگر پیاده‌سازی جدول نمادهای ما به این صورت باشد که هر یک از 
\lr{scope}
ها جدول نمادهای مربوط به خود را داشته باشند در این صورت پس از ورود به یک 
\lr{scope}
جدید او جدول نمادهای مربوط به 
\lr{scope}
جدید را درست خواهد کرد.
\فقره
اگر پیاده‌سازی جدول نمادها مانند فوق باشد پس خروج از جدول نمادها اشاره‌گر از جدول بالاتر به جدول نماد‌های مربوط به این
\lr{scope}
را پاک خواهد کرد و همچنین حافظه‌ی مربوطه را نیز از بین خواهد برد.
\پایان{شمارش}
\فقره
\فقره
از مزایای پیاده‌سازی با استفاده از جدول درهم‌سازی می‌توان به زمان بسیار پایین برای جست‌وجو و برای افزودن داده‌های جدید اشاره کرد ولی در عین حال می‌توان به معایب آن من جمله سختی در طراحی و نیاز به برنامه‌نویسی حرفه‌ای تر اشاره کرد. این روش معمولا در اکثر کامپایلرها مورد استفاده قرار می‌گیرد به این صورت که اسم متغیر کلید آن و باقی فیلدها به عنوان مقدار برگردانده شده توسط آن کلید است. در درخت جست‌و‌جوی دودویی زمان جست‌و‌جو و افزودن عناصر جدید بیشتر است ولی در عوض هزینه حافظه مصرفی آن پایین تر از جدول‌های درهم‌سازی می‌باشد.
\فقره
\شروع{شمارش}
\فقره
برای عبارت 
\lr{true or true and false}
دو درخت پارس مطابق شکل زیر وجود دارند:

\begin{LTR}

\Tree [.E [.E true ] or [.E [.E true ] and [.E false ] ] ] 
\Tree [.E [.E [.E true ] or [.E true ] ] and [.E false ] ]


\end{LTR}
\فقره
\پایان{شمارش}
\فقره 

این قانون بیان می‌کند که در انتخاب توکن‌ها باید حداکثر مقدار از ورودی مصرف شود، یعنی به طور مثال اگر دو قاعده‌ی منظم توانایی مطابقت با یک متن را دارند باید آنی که بیشترین طول از رشته را مصرف می‌کند انتخاب شود و یا اگر به طور مثال یک قاعده منظم می‌توان با پذیرفتن بخش کوچکتری از ورودی نیز با یک قاعده مطابقت پیدا کند و همچنین با پذیرفتن بخش بیشتری از این ورودی هم می‌تواند مطابقت پیدا کند باید با بخش بیشتر ورودی مطابقت پیدا کند.
\شروع{شمارش}
\فقره
این رشته هرگز وجود نخواهد داشت. اثبات با استفاده از برهان خلف. فرض کنید چنین رشته‌ای وجود داشته باشد. بنابراین در ابتدای آن یا دو حرف 
\lr{a}
آمده است و یا تعداد بیشتری حرف 
\lr{a}
آمده است. اگر حالت اول رخ دهد پس، پس از آن تعدادی 
\lr{b}
آمده است که نشان می‌دهد با انتخاب قانون ۱ عمل ما  عمل بهینه‌ای نبوده است چون با توجه به قانون 
\lr{Maximal Bunch}
قوانین ۲ یا ۳ قسمت‌های بیشتری از رشته را پوشش می‌دادند پس این حالت نمی‌توانسته رخ دهد. حال حالت دوم را بررسی می‌کنیم. اگر این رشته تعداد بیشتری 
\lr{a}
داشته باشد نیز به همین دلیل قوانین ۲ یا ۳ بخش بیشتری از رشته را پوشش می‌دهند پس این حالت نیز نمی‌توانسته است رخ داده باشد. بنابراین فرض خلف باطل و حکم برقرار است.
\فقره
این رشته می‌تواند خروجی موردنظر را تولید کند: 
\lr{bbaaaabaa}
\پایان{شمارش}
\پایان{شمارش}
\پایان{نوشتار}
